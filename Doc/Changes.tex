\newcommand{\Old}[1]{\begin{description}\item{Old :}#1}
\newcommand{\New}[1]{\item{New :}#1\end{description}}

%%
%% -*- NEW FEATURES -*-
%%
\section{Les nouveaut\'es}

\subsection{Les chaines de caracteres - Strings}
\Old{Quote et DoubleQuote sont equivalents.}
\New{Strings homogenes " " ou ' '.}

\subsection{Les Commentaires}
\Old{``//'' Comment la ligne jusqu'au EOL.}
\New{``//'' Comment la ligne jusqu'au EOL.
``/* */'' Commente une region entiere.}

\subsection{Les blocs functions}
\Old{Comportement divergeant suivant les cas
la declaration d'une nouvelle fonction pouvant
soit clore la precedente, soit etre imbriquee
dans celle en cours.}
\New{On force l'utilisation de ENDFUNCTION ce qui evite
ainsi toute ambiguite.}

\subsection{Les points de suspension}
\Old{Servent a continuer une expression sur plusieurs lignes
Y compris les ID, les NOMBRES et meme les mots clefs.}
\New{On ne coupe aucun token, seulement les instructions.
Les coupures se font avec 2 ou 3 points de suspension.}

\subsection{Les points et les nombres}
\Old{1./2}
\New{1 ./ 2 vs 1. / 2}

\subsection{Les declarations de matrices}
\Old{[,,,,,,a;;;;;;]}
\New{[a] ou [a;] Pas de syntaxes inutiles !!!}



%%
%% -*- FEATURES TO BE DISCUSSED -*-
%%
\section{Les fonctionnalites \`a inclure / exclure}

%%	!! TO BE AGREED !!
\subsection{Les declarations de matrices}
\Old{[a,,,b] $\simeq$ [a,b]}
\New{[a,,,b] $\simeq$ [a, nil, nil, b] Comme pour les appels de fonctions}


